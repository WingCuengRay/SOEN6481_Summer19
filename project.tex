\documentclass[10pt]{article}
\usepackage[utf8]{inputenc}
\usepackage{graphicx}
\usepackage{hyperref}
\usepackage{fontenc}
\usepackage{mathptmx}
\usepackage{geometry}
\usepackage{titling}
\setlength{\topskip}{0mm}
\setlength{\droptitle}{-8em} 
\title{{\large \textbf{CONCORDIA UNIVERSITY \\ DEPARTMENT OF COMPUTER SCIENCE AND SOFTWARE ENGINEERING \\ SOEN 6481: SOFTWARE SYSTEM REQUIREMENTS SPECIFICATION \\ SECTION WW WINTER 2019 \\ Irrational number: Champernowne Constant ($C_{10}$)}  \\ }}
\author{\normalsize \textbf {STUDENT NAME: YONGCONG LEI} \\ \normalsize \textbf{STUDENT IDENTIFICATION NUMBER: 40045701 }}
\date{}
\begin{document}
\maketitle

\section*{\normalsize \textbf{Characteristics}}
\paragraph{}
Champernowne Constant $C_{10}$ is a transcendental real constant whose decimal expansion has import properties. The constant is named because economist and mathematician D. G. Champernowne published it in 1933.

\paragraph{}
The first interesting property of Champernowne Constant is that the number is defined by combination of successive base-10 integers, which means $C_{10} = 0.12345678910111213141516…$.

\paragraph{}
Besides, it's also a transcendental numbers. All real transcendental numbers are irrational numbers, but not all irrational numbers are transcendental. The property of transcendental also makes Champernowne Constant uncountable infinite. However, $C_{10}$ is computable  although it is a transcendental number, which means that we can write a program to compute it with any desired precision.

\paragraph{}
What's more, Champernowne number was proved normal in base 10. The fact shows all of its digits being equally likely, All pairs of digits being equally likely, All triplets of digits being equally likely, etc. For example, we can find that string $[0], [1], ..., [9]$ are equally distributed in the constant at rate of 1/10, and $ [0,0],[0,1],...,[9,8],[9,9]$ are equally distributed at rate of 1/100.





\section*{\normalsize \textbf{Interview Questions:}}

\section*{\normalsize \textbf{Responses:}}

\section*{\normalsize \textbf{Analysis:}}

\section*{\normalsize \textbf{Problem 4 relevant concept, relevant relationship, problem domain model:}}

\section*{\normalsize \textbf{Problem 4 use case, use case model, normal scenario of use case:}}

\section*{\normalsize \textbf{TO BE DELETED}}
\begin{itemize}
\item The set of transcendental numbers is uncountably infinite. 
\item  No rational number is transcendental. All real transcendental numbers are irrational numbers. The converse is not true: not all irrational numbers are transcendental.
\item  Any non-constant algebraic function of a single variable yields a transcendental value when applied to a transcendental argument.
\item  An algebraic function of *several variables* may yield an algebraic number when applied to transcendental numbers if these numbers are not algebraically independent
\item \textit{\textbf{Cell phones instead of notes:}} for any two transcendental numbers a and b, at least one of a + b and ab must be transcendental.
\item The non-computable numbers are a strict subset of the transcendental numbers.


\item Champernowne Constant - For base 10, the number is defined by concatenating representations of successive integers.
\item Champernowne proved that $C_{10}$ is normal in base 10
\item Disjunctive sequence

\end{itemize}

\section*{\normalsize \textbf{References:}}
\url{1. https://en.wikipedia.org/wiki/Transcendental_number/} \\
\url{2. https://en.wikipedia.org/wiki/Champernowne_constantr/} 
\end{document}
