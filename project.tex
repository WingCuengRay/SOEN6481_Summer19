\documentclass[10pt]{article}
\usepackage[utf8]{inputenc}
\usepackage{graphicx}
\usepackage{hyperref}
\usepackage{fontenc}
\usepackage{mathptmx}
\usepackage{geometry}
\usepackage{titling}
\setlength{\topskip}{0mm}
\setlength{\droptitle}{-8em} 
\title{{\large \textbf{CONCORDIA UNIVERSITY \\ DEPARTMENT OF COMPUTER SCIENCE AND SOFTWARE ENGINEERING \\ SOEN 6481: SOFTWARE SYSTEM REQUIREMENTS SPECIFICATION \\ SECTION WW WINTER 2019 \\ Irrational number: Champernowne Constant ($C_{10}$)}  \\ }}
\author{\normalsize \textbf {STUDENT NAME: YONGCONG LEI} \\ \normalsize \textbf{STUDENT IDENTIFICATION NUMBER: 40045701 }}
\date{}
\begin{document}
\maketitle

\section{Characteristics}
\paragraph{}
Champernowne Constant $C_{10}$ is a transcendental real constant whose decimal expansion has import properties. The constant is named because economist and mathematician D. G. Champernowne published it in 1933.

\paragraph{}
The first interesting property of Champernowne Constant is that the number is defined by combination of successive base-10 integers, which means $C_{10} = 0.12345678910111213141516…$.

\paragraph{}
Besides, it's also a transcendental numbers. All real transcendental numbers are irrational numbers, but not all irrational numbers are transcendental. The property of transcendental also makes Champernowne Constant uncountable infinite. However, $C_{10}$ is computable  although it is a transcendental number, which means that we can write a program to compute it with any desired precision.

\paragraph{}
What's more, Champernowne number was proved normal in base 10. The fact shows all of its digits being equally likely, All pairs of digits being equally likely, All triplets of digits being equally likely, etc. For example, we can find that string $[0], [1], ..., [9]$ are equally distributed in the constant at rate of 1/10, and $ [0,0],[0,1],...,[9,8],[9,9]$ are equally distributed at rate of 1/100.

\pagebreak

\section{Interview}
My interviewee is my roommate. He has a background of mathematics, which might be useful for understanding the constant and coming up an idea of what the constant is about, why we need this constant and what kind of scenario in which the constant might be helpful.
\subsection{Questions}
\begin{enumerate}
    \item Are there any other Champernowne Constant with different bases?
    \item Do you think Champernowne Constant in useful in our programming?
    \item How can we get a transcendental numbers based on the Champernowne Constant?
    \item What's the property of Champernowne that might be useful in problem? 
    \item Please give an example of application that might use the Champernowne Constant.
\end{enumerate}
\subsection{Response}
\begin{enumerate}
    \item Yes. $C_{10}$ is only one of the Champernowne Constant in its family. As the equation shows, the constant is based on 10. There are other Champernowne Constant with other bases. For example, $C_{2}$ is a constant with binary sequence, while $C_16$ is a hexadecimal Champernowne constant.
    \item Not exactly. There is a pattern in Champernowne Constant, which makes it easy to get the constant in arbitrary length. In other words, it's easy to write a program to generate a Champernowne Constant of length L with base N. L is a number that is greater than or equal with 1 and N is number that is greater than or equal with 2.
    \item Actually, Any non-constant algebraic function of a single variable yields a transcendental value when applied to a transcendental argument. For example, we can get a new transcendental number by multiplying a constant value to $C_{10}$, which is $$\texttt{new transcendental number} = C_{10} * N \texttt { (N is a constant)}$$.
    \item The normality of Champernowne Constant might be useful in some practical problems. The property is interesting. It means if we number pick a digit in $C_{10}$, the possibility of picking a number from 0 to 10 is equal. If we pick two digits in $C_{10}$, the possibility of picking a number from 0 to 99 is equal. If we pick three digits in $C_{10}$, the possibility of picking a number from 0 to 999 is equal, etc.
    \item The interviewee didn't give an specific example for this question. He thinks although there are some interesting properties about Champernowne Constant, it's not really helpful in practice. Maybe we can make use of its normality on some algorithms which is used for random number generation. However, he was not clear about it.
\end{enumerate}
\subsection{Analysis}
By having the interview with my roommate, I have a better understanding of Champernowne Constant. As he said, although there are some interesting properties about Champernowne Constant, it's not really helpful in practice. It's just one of the pretty number that people (especially in mathematics field) thought would be good to experiment with. The pattern of the constant is clear since it is constructed in such a way that its (decimal) digits are easy to investigate.

\pagebreak

\section{Persona}

\section*{\normalsize \textbf{Problem 4 relevant concept, relevant relationship, problem domain model:}}

\section*{\normalsize \textbf{Problem 4 use case, use case model, normal scenario of use case:}}

\section*{\normalsize \textbf{TO BE DELETED}}
\begin{itemize}
\item The set of transcendental numbers is uncountably infinite. 
\item  No rational number is transcendental. All real transcendental numbers are irrational numbers. The converse is not true: not all irrational numbers are transcendental.
\item  Any non-constant algebraic function of a single variable yields a transcendental value when applied to a transcendental argument.
\item  An algebraic function of *several variables* may yield an algebraic number when applied to transcendental numbers if these numbers are not algebraically independent
\item \textit{\textbf{Cell phones instead of notes:}} for any two transcendental numbers a and b, at least one of a + b and ab must be transcendental.
\item The non-computable numbers are a strict subset of the transcendental numbers.


\item Champernowne Constant - For base 10, the number is defined by concatenating representations of successive integers.
\item Champernowne proved that $C_{10}$ is normal in base 10
\item Disjunctive sequence

\end{itemize}

\section*{\normalsize \textbf{References:}}
\url{1. https://en.wikipedia.org/wiki/Transcendental_number/} \\
\url{2. https://en.wikipedia.org/wiki/Champernowne_constantr/} 
\end{document}
